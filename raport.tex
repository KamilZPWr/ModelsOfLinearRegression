\documentclass[11pt,a4paper]{article}
\usepackage[utf8]{inputenc}
\usepackage{amsmath}
\usepackage{float}
\usepackage{amsfonts}
\usepackage{mathtools,amssymb,amsthm}
\usepackage{graphicx}
\usepackage{polski}
\mathtoolsset{showonlyrefs,mathic}
\author{Kamil Zawistowski, Michał Cerazy}


\begin{document}
\tableofcontents
\newpage
\section{Cel}
Celem prowadzonych badań jest odszukanie wyspępowania zależności między statystykami, w koszykówce, przy użyciu modeli regresji. 

\section{Zbiór danych}
Przedmiotem badań są statystyki poszczególnych zawodników występujących na parkietach NBA w sezonie 2014/2015. W omawianym okresie, w lidze wystąpiło 490 zawodników, co przekłada się na taką samą długość próbki. Polecenie wymaga próbki o długości co najmniej 500 pomiarów, jednak w związku z atrakcyjnością omawianych danych zdecydowaliśmy się na ich użycie. Dane pochodzą ze ........................ . 
\subsection{Opis zmiennych}
Zmienne kategoryczne wyróżnione są poprzez nawiasy zawierające wartości, które zmienna możep przyjąć. W sprawozdaniu zostały użyte następujące zmienne:
\begin{itemize}
	\item ,,Name'' -- imię oraz nazwisko,
	\item ''Age'' -- wiek,
	\item ,,Birth Place'' -- miejsce urodzenia (US, NONUS),
	\item ,,Height''-- wzrost,
	\item ,,Pos'' -- pozycja na boisku (PG, SG, C, PF, SF),
	\item ,,Team'' -- ostatni zespół, w którym grał zawodnik,
	\item ,,Weight'' -- waga,	
	\item ,,BMI'' -- wskaźnik Body Mass Indeks, 
	\item ,,Games Played'' - liczba rozegranych meczy,	
	\item ,,MIN'' -- liczba minut na boisku,
	\item ,,PTS'' -- liczba zdobytych punktów w sezonie,
	\item ,,FGM'' -- 
	\item ,,FGA'' -- 
	\item ,,FG p'' -- 
	\item ,,ThreePM'' -- 
	\item ,,3PA'' -- 
	\item ,,ThreeP p'' -- 
	\item ,,FTM'' -- 
	\item ,,FTA'' -- 
	\item ,,FT p'' -- 
	\item ,,OREB'' -- 	
	\item ,,DREB'' -- 	
	\item ,,REB'' -- liczba zbiórek,
	\item ,,AST'' -- liczba asyst,
	\item ,,STL'' -- liczba przejęć,
	\item ,,BLK'' -- liczba bloków,
	\item ,,TOV'' -- liczba strat,
	\item ,,PF'' --
	\item ,,EFF'' -- 
	\item ,,AST/TOV'' -- 
	\item ,,STL/TOV'' -- 
	\item ,,PPG'' -- liczba zdobytych punktów podzielona przez liczbę rozegranych meczy,
	\item ,,MPG'' -- liczba rozegranych minut podzielona przez liczbę rozegranych meczy.
\end{itemize}
\subsection{Wybór zmiennych zależnych i niezależnych}
	Zmienną zależną jest liczba punktów zdobytych w sezonie podzielona przez liczbę rozegranych meczy. Do zmiennych niezależnych należą !!!! UZUPEŁNIĆ ZMIENNE NIEZALEŻNE !!!!. Decydujemy się na taki wybór, gdyż chcemy poznać wpływ poszczególnych statystyk na liczbę zdobywanych punktów. (?) Odnalezienie odpowiedniego modelu będzie podstawą do wyznaczania zawodników pretendujących do bycia najlepiej punktującymi w kolejnych sezonach. (?)
	
	
\section{Dopasowanie modelu}
\subsection{Model uwzględniający cechy motoryczne zawodników}
Podjęliśmy trzy próby dobrania najlepszego modelu, który uzależniony jest od fizyczności zawodników. Oczywiście model musi być rozszerzony o ilość rozegranych minut na mecz, gdyż jest niezbędna informacja do analizy modelu.

\subsubsection{Model WWW}
Pierwszy z dopasowywanych modeli uwzględnia wzrost, wiek oraz wagę zawodników. Zmienne istotne dla dobranego model to minuty na mecz (jak zakładaliśmy wyżej, jest to najważniejsz czynnik) oraz waga. W tabeli \ref{model_www} przedstawione zostały wartości współczynników oraz pozostałe statystyki dla omawianego modelu. 

\begin{table}[H]
	\begin{tabular}{| c | c | c | c | c |}
		\hline
		Zmienna & Estymacja & Błąd standardowy & T-wartość & Pr$(<|t|)$\\ \hline
		Minuty na mecz & 0.54219 & 0.01242 & 43.638 & $<$2e-16\\ \hline
		Wzrost & -0.03394 & 0.02154 & -1.576 & 0.1156\\ \hline
		Wiek & -0.01660 & 0.02667 & -0.622 & 0.5340\\ \hline 
		Waga & 0.02721 & 0.01462 & 1.861 & 0.0634\\ \hline
		Stała & 1.63120 & 3.35589 & 0.486 & 0.6271 \\ \hline
	\end{tabular}
\caption{Model WWW - współczynniki modelu}
\label{model_www}
\end{table}
	
\subsubsection{Model WW}
Drugi z dopasowywanych modeli uwzględnia wiek oraz wagę zawodników. Odjęcie wzrostu ma na celu sprawdzenie jak zachowa się model po odjęciu jednej zmiennej nie będącej zmienną istotną. W tabeli \ref{model_ww} przedstawione zostały współczynniki modelu WW. Jak można zauważyć zmienne istotne dla modelu nie są już takie same, do minut na mecz dołączyła wartość stała. Ponadto waga zawodnika przestała należeć do istotnych zmiennych.

\begin{table}[H]
	\begin{tabular}{| c | c | c | c | c |}
		\hline
		Zmienna & Estymacja & Błąd standardowy & T-wartość & Pr$(<|t|)$\\ \hline
		Minuty na mecz & 0.54356 & 0.01241 & 43.78 & $<$2e-16\\ \hline
		Wiek & -0.01304 & 0.02661 & -0.490 & 0.62428 \\ \hline 
		Waga & 0.00852 & 0.00857 & 0.994 & 0.32059\\ \hline
		Stała & -3.35012 & 1.13018 & -2.964 & 0.00318\\ \hline
	\end{tabular}
	\caption{Model WW - współczynniki modelu}
	\label{model_ww}
\end{table}

\subsubsection{Model WB}
Trzeci z dopasowywanych modeli uwzględnia wiek oraz wskaźnik BMI. Omawiany wskaźnik jest popularną statystyką mówiącą o stanie zdrowia fizycznego i wyliczana jest na podstawie poniższego wzoru:
\begin{equation}
BMI = \frac{waga}{wzrost^2}.
\end{equation}
W związku z tym niejawnie do modelu włączamy wagę i wzrost zawodników ligi. Wyniki tego zabiegu przedstawione są w tabeli \ref{model_wb}. Podobnie jak w modelu WW istotne zmienne to liczba minut na boisku oraz stała. 
\begin{table}[H]
	\begin{tabular}{| c | c | c | c | c |}
		\hline
		Zmienna & Estymacja & Błąd standardowy & T-wartość & Pr$(<|t|)$\\ \hline
		Minuty na mecz & 0.54259 & 0.01238 & 43.838 & $<$2e-16\\ \hline
		Wiek & -0.01608 & 0.02667 & -0.603 & 0.54676\\ \hline 
		BMI & 0.09639 & 0.05878 & 1.640 & 0.10166\\ \hline  
		Stała & -4.84181 & 1.60703 & -3.013 & 0.00272\\ \hline	
	\end{tabular}
	\caption{Model WB - współczynniki modelu}
	\label{model_wb}
\end{table}
\subsubsection{Porównanie modeli}
Znając współczynniki dobranych modeli wyznaczyliśmy:
\begin{itemize}
	\item skorygowany współczynnik determinacji, oznaczany jako $R^2$,
	\item kryterium Akaikego, oznaczane jako AIC,
	\item logarytm wskaźnika wiarygodności, oznaczany jako LogLik,
	\item wartość w z testu Shapiro-Wilka, oznaczana jako W.
\end{itemize}

\begin{table}[H]
	\begin{center}
	\begin{tabular}{| c | c | c | c | c |}
		\hline
		Model & $R^2$ & AIC & LogLik & W\\ \hline
		WWW & 0.7977 & 2282.972 & -1135.486 & 0.95495\\ \hline
		WW & 0.7971 & 2283.475 & -1136.737 & 0.95403\\ \hline 
		WB & 0.7978 & 2281.766 & -1135.883 & 0.95493\\ \hline  
	\end{tabular}
	\caption{Porównanie modeli}
	\label{porownanie_modeli_w}
\end{center}
\end{table}
Każdy z dopasowanych modeli jest ,,poprawnym modelem'', gdyż ich residua mają rozkład normalny. Sprawdzenia tej hipotezy dokonaliśmy przy użyciu testu Shapiro-Wilka (długość naszej próbki nie przekracza 5000). Wartość W jest mniejsza od liczby 0.947 (wartość zaczerpnięta z tablic), co oznacza, że nie ma podstaw by odrzucić hipotezę o normalności residuów. W tabeli \ref{porownanie_modeli_w} widać, że model WB cechuje się najlepszym dopasowaniem do danych według współczynnika $R^2$. Kryterium informacyjne Akaikego również przemawia za wybraniem trzeciego modelu jako najlepiej opisującego badany zbiór wartości. Jedyną statystyką przemawiającą za innym modelem niż WB jest logarytm wskaźnik wiarygodności, który wskazuje, że to pierwszy model jest najlepszy. Na podstawie zaprezentowanych wyników należy wybrać model uwzględniający wiek oraz BMI zawodników.
\end{document}